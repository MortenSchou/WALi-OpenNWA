\section{The \texttt{wali::nwa::query} namespace}
\label{Se:namespace-query}

The \texttt{wali::nwa::query} namespace provides a large number of functions
for retrieving information about an \texttt{NWA}'s transitions. The questions
answered by these functions are of the form ``what are all symbols that
appear on a return transition where the call predecessor state is \texttt{s1}
and the return site state is \texttt{s1}?''

Functions return information either about all transitions or about
transitions of a particular type (internal, call, or return). The names of
these functions have regular patterns based on the information that is
known and desired, but the rules for producing the names are perhaps not as
simple as they could be. Because of this,
Tabs.~\ref{Ta:query-all-transitions}--\ref{Ta:query-return-transitions}
provide a quick-lookup of the functions.

The tables do not have \textsl{all} the information one needs to know in
order to call them, and the names use some shorthand; but they provide enough
information the specifics can be
looked up in the header or Doxygen documentation. The caption of each table
gives the header that contains the functions listed. In the interest of
space, the types of the function arguments are omitted, but they are likely to
be what you expect. (The number of arguments is also always sufficient to
uniquely identify the function since all types are just \texttt{wali::Key}s
anyway.) Almost all functions return a \texttt{StateSet}, \texttt{SymbolSet},
or a \texttt{std::set} of pairs of a state and a symbol. (Those that return a
set of pairs are marked specially.)

To use the table to answer the question above, we first look at the table for
return transitions (\tableref{Ta:query-return-transitions}). We know the call
predecessor and return site, so we find the corresponding row in the table
(fourth from the bottom). We want to know all the symbols, so we look at the
symbol column. We see there are two choices: \texttt{getReturnSym\_CallRet}
and \texttt{getExits}. By looking at \texttt{returns.hpp} or Doxygen
documentation, we can see that the former returns a \texttt{SymbolSet}, which
is what we want. The latter returns
a \texttt{std::set$<$std::pair$<$State,Symbol$>>$}, which contains the
information we desire, but is likely to be less convenient to
use. (The \texttt{State} component of each pair is the exit site, which is
where the function name \texttt{getExits} comes from.)





%\newcolumntype{x}[1]{p{#1}}
\newcommand{\RP}{\tnote{1}} %"returns pair"

\setlength{\extrarowheight}{4pt}

\newgeometry{bottom=0.75in,top=0.75in}

\begin{sidewaystable}\sffamily
\begin{threeparttable}
\begin{tabular}{p{1in}p{1in}|@{\hspace{0.1in}}p{2.2in}p{2.2in}p{2in}}
\toprule\toprule
\multicolumn{2}{c|@{\hspace{0.1in}}}{\textsl{What you know}} & \multicolumn{3}{c}{\textsl{What you want}}                                                                 \tabularnewline
 this...        & ... and this      &    sources                    &   symbols                          &    targets                     \tabularnewline
\midrule
\midrule %-------------------------------------------------------------------------------------------------------------------------------
  source        &  (nothing)        &      ---                      &  (none)                            &  getSuccessors(nwa, src)       \tabularnewline
                &  symbol           &      ---                      &        ---                         &  getSuccessors(nwa, src, sym)  \tabularnewline
                &  target           &      ---                      &  getSymbol(nwa, src, tgt, \&sym)   &   ---                          \tabularnewline
\midrule %-------------------------------------------------------------------------------------------------------------------------------
  symbol        &  (nothing)        &                               &        ---                         &   (none)                       \tabularnewline
                &  target           & getPredecessors(nwa, sym, tgt)&       ---                         &   ---                          \tabularnewline
\midrule %-------------------------------------------------------------------------------------------------------------------------------
  target        &  (nothing)        & getPredecessors(nwa, tgt)     &  (none)                            &   ---                          \tabularnewline
\bottomrule\bottomrule
\end{tabular}
\caption{Query functions for all transitions. These functions are
  in the namespace \texttt{wali::nwa::query}; include the
  file \texttt{wali/nwa/query/transitions.hpp}. For return transitions, the
  ``source'' is the first component of the transition; nothing involving call
  predecessors (the second component) appears in this sidewaystable. A table
  entry of ``---'' means that square does not make sense. }
\end{threeparttable}
\label{Ta:query-all-transitions}
\end{sidewaystable}

\begin{sidewaystable}\sffamily
\begin{threeparttable}
\begin{tabular}{p{1in}p{1in}|@{\hspace{0.1in}}p{2.2in}p{2.2in}p{2in}}
\toprule\toprule
\multicolumn{2}{c|@{\hspace{0.1in}}}{\textsl{What you know}} & \multicolumn{3}{c}{\textsl{What you want}}                                                                 \tabularnewline
 this...        & ... and this      &    sources                    &   symbols                          &    targets                     \tabularnewline
\midrule
\midrule %-------------------------------------------------------------------------------------------------------------------------------
 (nothing)      &  (nothing)        & getSources(nwa)               &  getInternalSym(nwa)               &  getTargets(nwa)               \tabularnewline
\midrule %-------------------------------------------------------------------------------------------------------------------------------
  source        &  (nothing)        &      ---                      &  getInternalSym\_Source(nwa, src) \newline
                                                                       or getTargets(nwa, src)\RP        &  getTargets(nwa, src)\RP       \tabularnewline
                &  symbol           &      ---                      &        ---                         &  getTargets(nwa, src, sym)     \tabularnewline
                &  target           &      ---                      &  getInternalSym(nwa, src, tgt)     &   ---                          \tabularnewline
\midrule %-------------------------------------------------------------------------------------------------------------------------------
  symbol        &  (nothing)        & getSources\_Sym(nwa, sym)     &        ---                         &  getTargets\_Sym(nwa, sym)     \tabularnewline
                &  target           & getSources(nwa, sym, tgt)     &        ---                         &   ---                          \tabularnewline
\midrule %-------------------------------------------------------------------------------------------------------------------------------
  target        &  (nothing)        & getSources(nwa, tgt)\RP       &  getInternalSym\_Target(nwa, tgt) \newline
                                                                       or getSources(nwa, tgt)\RP        &   ---                          \tabularnewline
\bottomrule\bottomrule
\end{tabular}
\begin{tablenotes}
  \item[1] Returns a set of pairs (either source/symbol or symbol/target).
\end{tablenotes}
\caption{Query functions for internal transitions. 
  These functions are in the namespace \texttt{wali::nwa::query};
  include the file \texttt{wali/nwa/query/internals.hpp}.
  A table entry of ``---'' means that square does not make
  sense.}
\label{Ta:query-internal-transitions}
\end{threeparttable}
\end{sidewaystable}

\begin{sidewaystable}\sffamily
\begin{threeparttable}
\begin{tabular}{p{1in}p{1in}|@{\hspace{0.1in}}p{2.2in}p{2.2in}p{2in}}
\toprule\toprule
\multicolumn{2}{c|@{\hspace{0.1in}}}{\textsl{What you know}} & \multicolumn{3}{c}{\textsl{What you want}}                                                                 \tabularnewline
 this...        & ... and this      &    call sites                   &   symbols                          &    entries                     \tabularnewline
\midrule
\midrule %-------------------------------------------------------------------------------------------------------------------------------
 (nothing)      &  (nothing)        & getCallSites(nwa)               &  getCallSym(nwa)                   &  getEntries(nwa)               \tabularnewline
\midrule %-------------------------------------------------------------------------------------------------------------------------------
 call site      &  (nothing)        &      ---                        &  getCallSym\_Call(nwa, call)\newline
                                                                         or getEntries(nwa, call)\RP       &  getEntries(nwa, call)\RP      \tabularnewline
                &  symbol           &      ---                        &        ---                         &  getEntries(nwa, call, sym)    \tabularnewline
                &  target           &      ---                        &  getCallSym(nwa, call, ent)        &   ---                          \tabularnewline
\midrule %-------------------------------------------------------------------------------------------------------------------------------
  symbol        &  (nothing)        & getCallSites\_Sym(nwa, sym)     &        ---                         &  getEntries\_Sym(nwa, sym)     \tabularnewline
                &  target           & getCallSites(nwa, sym, ent)     &        ---                         &   ---                          \tabularnewline
\midrule %-------------------------------------------------------------------------------------------------------------------------------
  entry         &  (nothing)        & getCallSites(nwa, ent)\RP       &  getCallSym\_Entry(nwa, ent) \newline
                                                                         or getCallSites(nwa, ent)\RP      &   ---                          \tabularnewline
\bottomrule\bottomrule
\end{tabular}
\begin{tablenotes}
  \item[1] Returns a set of pairs (either call site/symbol or symbol/entry).
\end{tablenotes}
\caption{Query functions for call transitions. These functions are in the
  namespace \texttt{wali::nwa::query}; include the
  file \texttt{wali/nwa/query/calls.hpp}. The ``call site'' is the source of the transition (and uses the argument 
  name \texttt{call}), and the ``entry'' of the transition is the target (and
  uses the argument name \texttt{ent}). }
\label{Ta:query-call-transitions}
\end{threeparttable}
\end{sidewaystable}

\begin{sidewaystable}\footnotesize\sffamily
\begin{threeparttable}
\begin{tabular}{p{0.6in}p{0.65in}p{0.6in}|@{\hspace{0.1in}}p{1.75in}p{1.9in}p{1.9in}p{2in}}
\toprule\toprule
\multicolumn{3}{c|@{\hspace{0.1in}}}{\textsl{What you know}}  & \multicolumn{4}{c}{\textsl{What you want}}                                                                                                                                                \tabularnewline
 this...  & and this ...  & and this & exit sites & call predecessors &
 symbols & return sites \tabularnewline
\midrule
\midrule %--------------------------------------------------------------------------------------------------------------------------------------------------------------------------------------------------------------------
 (nothing)      &  (nothing)        &  (nothing)    & getExits(nwa)                 &  getCalls(nwa)                        &  getReturnSym(nwa)                           &  getReturns(nwa)                            \tabularnewline
\midrule %--------------------------------------------------------------------------------------------------------------------------------------------------------------------------------------------------------------------
 exit site      &  (nothing)        &  (nothing)    &      ---                      &  getCalls\_Exit(nwa, exit)\RP         &  getReturnSym\_Exit(nwa, exit) \newline
                                                                                                                               or getReturns\_Exit(nwa, exit)\RP\newline
                                                                                                                               or getCalls\_Exit(nwa, exit)\RP             &  getReturns\_Exit(nwa, exit)\RP             \tabularnewline
                \cline{2-7} %-------------------------------------------------------------------------------------------------------------------------------------------------------------------------------------------------
                &  call pred        &  (nothing)    &      ---                      &    ---                                &  getReturnSym\_ExitCall(nwa, exit, \newline
                                                                                                                               \phantom{getReturnSym\_ExitCall(}call) \newline
                                                                                                                               or getReturns(nwa, exit, call)\RP           &  getReturns(nwa, exit, call)\RP             \tabularnewline
                &                   &  symbol       &      ---                      &    ---                                &        ---                                   &  getReturns(nwa, exit, call, sym)           \tabularnewline
                &                   &  return       &      ---                      &    ---                                &  getReturnSym(nwa, exit, call, \newline
                                                                                                                               \phantom{getReturnSym(}ret)                 &    ---                                      \tabularnewline
                \cline{2-7} %-------------------------------------------------------------------------------------------------------------------------------------------------------------------------------------------------
                &  symbol           &  (nothing)    &      ---                      &  getCalls\_Exit(nwa, exit, sym)       &        ---                                   &  getReturns\_Exit(nwa, exit, sym)           \tabularnewline
                &                   &  return       &      ---                      &  getCalls(nwa, exit, sym, ret)        &        ---                                   &  getEntries(nwa, call, sym, ret)            \tabularnewline
                \cline{2-7} %-------------------------------------------------------------------------------------------------------------------------------------------------------------------------------------------------
                &  return site      &  (nothing)    &      ---                      &  getCalls(nwa, exit, ret)\RP          &  getReturnSym\_ExitRet(nwa, exit, \newline
                                                                                                                               \phantom{getReturnSym\_ExitRet(}ret) \newline
                                                                                                                               or getCalls(nwa, exit, ret)\RP              &   ---                                       \tabularnewline
\midrule %-------------------------------------------------------------------------------------------------------------------------------------------------------------------------------------------------------------------
 call pred      &  (nothing)        &  (nothing)    & getExits\_Call(nwa, call)\RP  &   ---                                 &  getReturnSym\_Call(nwa, call) \newline
                                                                                                                               or getReturns\_Call(nwa, call)\RP\newline
                                                                                                                               or getExits\_Call(nwa, call)\RP            &  getReturnSites(nwa, call) or \newline
                                                                                                                                                                             getCallSuccessors(nwa, call) or \newline
                                                                                                                                                                             getReturns\_Call(nwa, call)\RP              \tabularnewline
                \cline{2-7} %-------------------------------------------------------------------------------------------------------------------------------------------------------------------------------------------------
                &  symbol           &  (nothing)    & getExits\_Call(nwa, call, sym)&   ---                                 &        ---                                   &  getCallSuccessors(nwa, call, sym) \newline
                                                                                                                                                                              or getReturns\_Call(nwa, call, sym)        \tabularnewline
                &                   &  return       & getExits(nwa, call, sym, ret) &   ---                                 &        ---                                   &    ---                                      \tabularnewline
                \cline{2-7} %-------------------------------------------------------------------------------------------------------------------------------------------------------------------------------------------------
                &  return site      &  (nothing)    & getExits(nwa, call, ret)\RP   &   ---                                 &  getReturnSym\_CallRet(nwa, call, \newline
                                                                                                                               \phantom{getReturnSym\_CallRet(}ret) \newline
                                                                                                                               or getExits(nwa, call, ret)\RP              &   ---                                       \tabularnewline
\midrule %--------------------------------------------------------------------------------------------------------------------------------------------------------------------------------------------------------------------
 symbol         &  (nothing)        &  (nothing)    & getExits\_Sym(nwa, c)         &  getCalls\_Sym(nwa, c)                &   ---                                        &  getReturns\_Sym(nwa, sym)                  \tabularnewline
                \cline{2-7} %-------------------------------------------------------------------------------------------------------------------------------------------------------------------------------------------------
                &  return site      &  (nothing)    & getExits\_Ret(nwa, call, ret) &  getCallPredecessors(nwa, sym, ret) \newline
                                                                                       or getCalls\_Ret(nwa, sym, c)        &   ---                                        &   ---                                       \tabularnewline
\midrule %--------------------------------------------------------------------------------------------------------------------------------------------------------------------------------------------------------------------
 return site    &  (nothing)         & (nothing)    & getExits\_Ret(nwa, ret)       &  getCallPredecessors(nwa, ret) \newline
                                                                                       or getCalls\_Ret(nwa, ret)\RP        &  getReturnSym\_Ret(nwa, ret) or \newline
                                                                                                                               getCalls\_Ret(nwa, ret)\RP                  &   ---                                       \tabularnewline
\bottomrule\bottomrule
\end{tabular}
\begin{tablenotes}
  \item[1] Returns a set of pairs (a symbol with one of the states, in the order of the raw transition).
\end{tablenotes}
\caption{Query functions for call transitions. These functions are in the
  namespace \texttt{wali::nwa::query}; include the
  file \texttt{wali/nwa/query/returns.hpp}.   The ``exit site'' is the source of the transition
  (the first component) and uses the argument name \texttt{x} in this table;
  the ``call predecessor'' is the second component and uses the argument
  name \texttt{c}; the symbol is the third component uses the argument
  name \texttt{s}; the ``return site'' is the fourth component and uses the
  argument name \texttt{r}. }
\label{Ta:query-return-transitions}
\end{threeparttable}
\end{sidewaystable}
\restoregeometry
