%\newcolumntype{x}[1]{p{#1}}
\newcommand{\RP}{\tnote{1}} %"returns pair"

\setlength{\extrarowheight}{4pt}

\newgeometry{bottom=0.75in,top=0.75in}

\begin{sidewaystable}\sffamily
\begin{threeparttable}
\begin{tabular}{p{1in}p{1in}|@{\hspace{0.1in}}p{2.2in}p{2.2in}p{2in}}
\toprule\toprule
\multicolumn{2}{c|@{\hspace{0.1in}}}{\textsl{What you know}} & \multicolumn{3}{c}{\textsl{What you want}}                                                                 \tabularnewline
 this...        & ... and this      &    sources                    &   symbols                          &    targets                     \tabularnewline
\midrule
\midrule %-------------------------------------------------------------------------------------------------------------------------------
  source        &  (nothing)        &      ---                      &  (none)                            &  getSuccessors(Key src)           \tabularnewline
                &  symbol           &      ---                      &        ---                         &  getSuccessors(Key src, Key sym)  \tabularnewline
                &  target           &      ---                      &  getSymbol(Key src, Key tgt,\newline
                                                                       \phantom{getSymbol(}Key\& sym)    &   ---                          \tabularnewline
\midrule %-------------------------------------------------------------------------------------------------------------------------------
  symbol        &  (nothing)        &                               &        ---                         &   (none)                       \tabularnewline
                &  target           & getPredecessors(Key sym, Key tgt) &    ---                         &   ---                          \tabularnewline
\midrule %-------------------------------------------------------------------------------------------------------------------------------
  target        &  (nothing)        & getPredecessors(Key tgt)      &  (none)                            &   ---                          \tabularnewline
\bottomrule\bottomrule
\end{tabular}
\caption{Query functions for all transitions. For return transitions, the
  ``source'' is the first component of the transition; nothing involving call
  predecessors (the second component) appears in this sidewaystable. A table
  entry of ``---'' means that square does not make sense. These functions are
  in the namespace \texttt{wali::nwa::query}; include the
  file \texttt{wali/nwa/query/transitions.hpp}.}
\end{threeparttable}
\label{Ta:query-all-transitions}
\end{sidewaystable}

\begin{sidewaystable}\sffamily
\begin{threeparttable}
\begin{tabular}{p{1in}p{1in}|@{\hspace{0.1in}}p{2.2in}p{2.2in}p{2in}}
\toprule\toprule
\multicolumn{2}{c|@{\hspace{0.1in}}}{\textsl{What you know}} & \multicolumn{3}{c}{\textsl{What you want}}                                                                 \tabularnewline
 this...        & ... and this      &    sources                    &   symbols                          &    targets                     \tabularnewline
\midrule
\midrule %-------------------------------------------------------------------------------------------------------------------------------
 (nothing)      &  (nothing)        & getSources()                  &  getInternalSym()                  &  getTargets()                  \tabularnewline
\midrule %-------------------------------------------------------------------------------------------------------------------------------
  source        &  (nothing)        &      ---                      &  getInternalSym\_Source(Key src) \newline
                                                                       or getTargets(Key src)\RP         &  getTargets(Key src)\RP        \tabularnewline
                &  symbol           &      ---                      &        ---                         &  getTargets(Key src, Key sym)  \tabularnewline
                &  target           &      ---                      &  getInternalSym(Key src, Key tgt)  &   ---                          \tabularnewline
\midrule %-------------------------------------------------------------------------------------------------------------------------------
  symbol        &  (nothing)        & getSources\_Sym(Key sym)      &        ---                         &  getTargets\_Sym(Key sym)      \tabularnewline
                &  target           & getSources(Key sym, Key tgt)  &        ---                         &   ---                          \tabularnewline
\midrule %-------------------------------------------------------------------------------------------------------------------------------
  target        &  (nothing)        & getSources(Key tgt)\RP        &  getInternalSym\_Target(Key tgt) \newline
                                                                       or getSources(Key tgt)\RP         &   ---                          \tabularnewline
\bottomrule\bottomrule
\end{tabular}
\begin{tablenotes}
  \item[1] Returns a set of pairs (either source/symbol or symbol/target).
\end{tablenotes}
\caption{Query functions for internal transitions. A table entry of ``---'' means that square doesn't make
  sense. These functions are in the namespace \texttt{wali::nwa::query};
  include the file \texttt{wali/nwa/query/internals.hpp}.}
\end{threeparttable}
\end{sidewaystable}

\begin{sidewaystable}\sffamily
\begin{threeparttable}
\begin{tabular}{p{1in}p{1in}|@{\hspace{0.1in}}p{2.2in}p{2.2in}p{2in}}
\toprule\toprule
\multicolumn{2}{c|@{\hspace{0.1in}}}{\textsl{What you know}} & \multicolumn{3}{c}{\textsl{What you want}}                                                                 \tabularnewline
 this...        & ... and this      &    call sites                   &   symbols                          &    entries                     \tabularnewline
\midrule
\midrule %-------------------------------------------------------------------------------------------------------------------------------
 (nothing)      &  (nothing)        & getCallSites()                  &  getCallSym()                      &  getEntries()                  \tabularnewline
\midrule %-------------------------------------------------------------------------------------------------------------------------------
 call site      &  (nothing)        &      ---                        &  getCallSym\_Call(Key call) \newline
                                                                         or getEntries(Key call)\RP        &  getEntries(Key call)\RP       \tabularnewline
                &  symbol           &      ---                        &        ---                         &  getEntries(Key call, Key sym) \tabularnewline
                &  target           &      ---                        &  getCallSym(Key call, Key ent)     &   ---                          \tabularnewline
\midrule %-------------------------------------------------------------------------------------------------------------------------------
  symbol        &  (nothing)        & getCallSites\_Sym(Key sym)      &        ---                         &  getEntries\_Sym(Key sym)      \tabularnewline
                &  target           & getCallSites(Key sym, Key ent)  &        ---                         &   ---                          \tabularnewline
\midrule %-------------------------------------------------------------------------------------------------------------------------------
  entry         &  (nothing)        & getCallSites(Key ent)\RP        &  getCallSym\_Entry(Key ent) \newline
                                                                         or getCallSites(Key ent)\RP       &   ---                          \tabularnewline
\bottomrule\bottomrule
\end{tabular}
\begin{tablenotes}
  \item[1] Returns a set of pairs (either call site/symbol or symbol/entry).
\end{tablenotes}
\caption{Query functions for call transitions. The ``call site'' is the source of the transition (and uses the argument 
  name \texttt{call}), and the ``entry'' of the transition is the target (and
  uses the argument name \texttt{ent}). These functions are in the
  namespace \texttt{wali::nwa::query}; include the
  file \texttt{wali/nwa/query/calls.hpp}.}
\end{threeparttable}
\end{sidewaystable}

\begin{sidewaystable}\footnotesize\sffamily
\begin{threeparttable}
\begin{tabular}{p{0.6in}p{0.65in}p{0.6in}|@{\hspace{0.1in}}p{1.75in}p{1.9in}p{1.9in}p{2in}}
\toprule\toprule
\multicolumn{3}{c|@{\hspace{0.1in}}}{\textsl{What you know}}  & \multicolumn{4}{c}{\textsl{What you want}}                                                                                                                                                \tabularnewline
 this...  & and this ...  & and this & exit sites & call predecessors &
 symbols & return sites \tabularnewline
\midrule
\midrule %--------------------------------------------------------------------------------------------------------------------------------------------------------------------------------------------------------------------
 (nothing)      &  (nothing)        &  (nothing)    & getExits()                    &  getCalls()                           &  getReturnSym()                            &  getReturns()                               \tabularnewline
\midrule %--------------------------------------------------------------------------------------------------------------------------------------------------------------------------------------------------------------------
 exit site      &  (nothing)        &  (nothing)    &      ---                      &  getCalls\_Exit(Key x)\RP             &  getReturnSym\_Exit(Key x) or \newline
                                                                                                                               getReturns\_Exit(Key x)\RP\ or \newline
                                                                                                                               getCalls\_Exit(Key x)\RP                  &  getReturns\_Exit(Key x)\RP                 \tabularnewline
                \cline{2-7} %-------------------------------------------------------------------------------------------------------------------------------------------------------------------------------------------------
                &  call pred        &  (nothing)    &      ---                      &    ---                                &  getReturnSym\_ExitCall(Key x, \newline
                                                                                                                               \phantom{getReturnSym\_ExitCall(}Key c) \newline
                                                                                                                               or getReturns(Key x, Key c)\RP            &  getReturns(Key call, Key c)\RP             \tabularnewline
                &                   &  symbol       &      ---                      &    ---                                &        ---                                 &  getReturns(Key call, Key c, Key s)         \tabularnewline
                &                   &  return       &      ---                      &    ---                                &  getReturnSym(Key x, Key call, \newline
                                                                                                                               \phantom{getReturnSym(}Key s)             &    ---                                      \tabularnewline
                \cline{2-7} %-------------------------------------------------------------------------------------------------------------------------------------------------------------------------------------------------
                &  symbol           &  (nothing)    &      ---                      &  getCalls\_Exit(Key x, Key s)         &        ---                                 &  getReturns\_Exit(Key x, Key s              \tabularnewline
                &                   &  return       &      ---                      &  getCalls(Key x, Key s, Key r)        &        ---                                 &  getEntries(Key call, Key s)                \tabularnewline
                \cline{2-7} %-------------------------------------------------------------------------------------------------------------------------------------------------------------------------------------------------
                &  return site      &  (nothing)    &      ---                      &  getCalls(Key x, Key r)\RP            &  getReturnSym\_ExitRet(Key x, \newline
                                                                                                                               \phantom{getReturnSym\_ExitRet(}Key r) \newline
                                                                                                                               or getCalls(Key x, Key r)\RP              &   ---                                       \tabularnewline
\midrule %-------------------------------------------------------------------------------------------------------------------------------------------------------------------------------------------------------------------
 call pred      &  (nothing)        &  (nothing)    & getExits\_Call(Key c)\RP      &   ---                                 &  getReturnSym\_Call(Key c) or \newline
                                                                                                                               getReturns\_Call(Key c)\RP\ or \newline
                                                                                                                               getExits\_Call(Key c)\RP                  &  getReturnSites(Key c) or \newline
                                                                                                                                                                            getCallSuccessors(Key c) or \newline
                                                                                                                                                                            getReturns\_Call(Key c)\RP                 \tabularnewline
                \cline{2-7} %-------------------------------------------------------------------------------------------------------------------------------------------------------------------------------------------------
                &  symbol           &  (nothing)    & getExits\_Call(Key c, Key s)  &   ---                                 &        ---                                 &  getCallSuccessors(Key c, Key s) \newline
                                                                                                                                                                            or getReturns\_Call(Key c, Key s)          \tabularnewline
                &                   &  return       & getExits(Key c, Key s, Key r) &   ---                                 &        ---                                 &    ---                                      \tabularnewline
                \cline{2-7} %-------------------------------------------------------------------------------------------------------------------------------------------------------------------------------------------------
                &  return site      &  (nothing)    & getExits(Key c, Key r)\RP     &   ---                                 &  getReturnSym\_CallRet(Key c, \newline
                                                                                                                               \phantom{getReturnSym\_CallRet(}Key r) \newline
                                                                                                                               or getExits(Key c, Key r)\RP              &   ---                                       \tabularnewline
\midrule %--------------------------------------------------------------------------------------------------------------------------------------------------------------------------------------------------------------------
 symbol         &  (nothing)        &  (nothing)    & getExits\_Sym(Key c)          &  getCalls\_Sym(Key c)                 &   ---                                      &  getReturns\_Sym(Key s)                     \tabularnewline
                \cline{2-7} %-------------------------------------------------------------------------------------------------------------------------------------------------------------------------------------------------
                &  return site      &  (nothing)    & getExits\_Ret(Key c, Key r)   &  getCallPredecessors(Key s, Key r) \newline
                                                                                       or getCalls\_Ret(Key s, Key c)       &   ---                                      &   ---                                       \tabularnewline
\midrule %--------------------------------------------------------------------------------------------------------------------------------------------------------------------------------------------------------------------
 return site    &  (nothing)         & (nothing)    & getExits\_Ret(Key r)          &  getCallPredecessors(Key r) \newline
                                                                                       or getCalls\_Ret(Key r)\RP           &  getReturnSym\_Ret(Key r) or \newline
                                                                                                                               getCalls\_Ret(Key r)\RP                   &   ---                                       \tabularnewline
\bottomrule\bottomrule
\end{tabular}
\begin{tablenotes}
  \item[1] Returns a set of pairs (a symbol with one of the states, in the order of the raw transition).
\end{tablenotes}
\caption{Query functions for call transitions. The ``exit site'' is the source of the transition
  (the first component) and uses the argument name \texttt{x} in this table;
  the ``call predecessor'' is the second component and uses the argument
  name \texttt{c}; the symbol is the third component uses the argument
  name \texttt{s}; the ``return site'' is the fourth component and uses the
  argument name \texttt{r}.  These functions are in the
  namespace \texttt{wali::nwa::query}; include the
  file \texttt{wali/nwa/query/returns.hpp}.}
\end{threeparttable}
\end{sidewaystable}
\restoregeometry
