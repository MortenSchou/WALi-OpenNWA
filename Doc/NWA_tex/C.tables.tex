\section{Tables}
\label{App:Tables}
This section provides a number of quick-reference tables.

The tables do not have \textsl{all} the information one needs to know in
order to call the functions, and the entries use some shorthands; but they
provide enough information for specifics to be looked up in the corresponding
header or Doxygen documentation. The caption of each table gives the header
that contains the functions listed. In the interest of space, the types of
the function arguments are sometimes omitted, but they are likely to be what you
expect.


\newgeometry{bottom=0.70in,top=0.70in}
\begin{sidewaystable}\sffamily\footnotesize
\begin{threeparttable}
\caption{\vbox{Table of functions for accessing and manipulating elements
  of NWAs. All are members of the \texttt{NWA} class, and thus declared in \texttt{wali/nwa/NWA.hpp}. \TODO{Fix footnote
    numbering}}
}
\label{Ta:simple-functions}
\begin{tabular}{p{0.75in}p{1.55in}p{1.6in}p{1.35in}p{1.27in}p{1.2in}p{1.5in}}
\toprule
                 &  add\tnote{1}                            &  remove\tnote{1}                          &  check membership\tnote{3} &  count                &  clear                     &  get\tnote{4}                 \\
\hline\hline %---------------------------------------------------------------------------------------------------------------------------------------------------------------------------------------
states           &  addState(State st)                      &  removeState(State st)\tnote{5}           &  isState(State st)         &  sizeStates()         &  clearStates()\tnote{5}    &  getStates() or \newline
                                                                                                                                                                                             \{begin,end\}States()        \\
initial \newline
states           &  addInitialState(State st)               &  removeInitialState(State st)             &  isInitialState(State st)  &  sizeInitialStates()  &  clearInitialStates()      &  getInitialStates() or \newline
                                                                                                                                                                                             \{begin,end\}InitialStates() \\
final
\newline  states &  addFinalState(State st)                 &  removeFinalState(State st)               &  isFinalState(State)       &  sizeFinalStates()    &  clearFinalStates()        &  getFinalStates() or \newline
                                                                                                                                                                                             \{begin,end\}FinalStates()    \\
\hline %---------------------------------------------------------------------------------------------------------------------------------------------------------------------------------------
symbols\tnote{2} &  addSymbol(Symbol sym)                   &  removeSymbol(Symbol sym)\tnote{5}        &  isSymbol(Symbol sym)      &  sizeSymbols()        &  clearSymbols()            &  getSymbols() or \newline
                                                                                                                                                                                             \{begin,end\}Symbols()         \\
\hline %---------------------------------------------------------------------------------------------------------------------------------------------------------------------------------------
all \newline
transitions      &  ---                                     &  findTrans(State s1, \newline
                                                               \phantom{find}Symbol sym, State s2)&  ---                       &  sizeTrans()          &  clearTrans()              &  ---                           \\
internal \newline
transitions      &  addInternalTrans(\phantom{.}...\tnote{6}\phantom{a})
                                                            &  removeInternalTrans(\phantom{.}...\tnote{6}\phantom{a})
                                                                                                        &  ---                       &  sizeInternalTrans()  &  ---                       &  \{begin,end\}internalTrans()  \\
call \newline
transitions      &  addCallTrans(\phantom{.}...\tnote{6}\phantom{a})
                                                            &  removeCallTrans(\phantom{.}...\tnote{6}\phantom{a})
                                                                                                        &  ---                       &  sizeCallTrans()      &  ---                       &  \{begin,end\}callTrans()      \\
return \newline
transitions      &  addReturnTrans(\phantom{.}...\tnote{6}\phantom{a})
                                                            &  removeReturnTrans(\phantom{.}...\tnote{6}\phantom{a})
                                                                                                        &  ---                       &  sizeReturnTrans()  &  ---                       &  \{begin,end\}returnTrans()\\
\hline
\end{tabular}
\begin{tablenotes}
  \item[1] These functions return a \texttt{bool} indicating whether the item
    was added/removed. Adding a transition implicitly adds all
    states and symbols in that transition if they are not already present.
    Removing a state or symbol removes all transitions the removed item was a
    part of.
  \item[2] \texttt{WALI\_EPSILON} and \texttt{WALI\_WILD} are not explicit
    members of the symbol set. This has the following consequences (for
    \texttt{s} as epsilon or wild): \texttt{addSymbol(s)} and
    \texttt{removeSymbol(s)} are both no-ops and return \texttt{false},
    \texttt{isSymbol(s)} returns false, \texttt{sizeSymbols()} does not count
    epsilon or wild, and neither the set returned by \texttt{getSymbols()}
    nor the iterator range \texttt{\{begin,end\}Symbols()} will contain
    epsilon or wild.
  \item[3] These functions return a \texttt{bool} with the natural
    interpretation.
  \item[4] The entries of the form, e.g., ``\{begin,end\}States()'' denote a pair
    of functions, each of which returns an iterator. The type of that iterator
    is either a \texttt{NWA::StateIterator},
    \texttt{NWA::SymbolIterator}, \texttt{NWA::CallIterator},
    \texttt{NWA::InternalIterator}, or \texttt{NWA::ReturnIterator}, as appropriate; all of these are
    actually const iterators. The other functions (``get\textit{Kind}()'')
    return a \texttt{StateSet} or \texttt{SymbolSet} as appropriate.
  \item[5] Removing a state or a symbol also removes all transitions
    involving it. (Hence clearing all states or clearing all symbols also
    clears all transitions.)
  \item[6] There are two overloads of each of these functions. The first
    takes each element of the transition tuple individually,
    e.g., \texttt{addCallTrans(State src, State sym, State tgt)}. The
    second takes a (constant reference to an) \texttt{NWA::Internal},
    \texttt{NWA::Call}, or \texttt{NWA::Return} object (as
    appropriate). (These are typedefs of a \texttt{Triple} or \texttt{Quad}
    of the appropriate type.)
\end{tablenotes}
\end{threeparttable}
\end{sidewaystable}
\restoregeometry



\newcommand{\RP}{\tnote{1}} %"returns pair"

\setlength{\extrarowheight}{4pt}

\newgeometry{bottom=0.75in,top=0.75in}
\begin{sidewaystable}\sffamily
\begin{threeparttable}
\caption{\textbf{Query functions for all transitions.} These functions are
  in the namespace \texttt{wali::nwa::query}; include the
  file \texttt{wali/nwa/query/transitions.hpp}. For return transitions, the
  ``source'' is the first component of the transition; nothing involving call
  predecessors (the second component of return transitions) appears in this table. A table
  entry of ``---'' means that the combination of arguments does not make sense. }
\begin{tabular}{p{1in}p{1in}|@{\hspace{0.1in}}p{2.2in}p{2.2in}p{2in}}
\toprule\toprule
\multicolumn{2}{c|@{\hspace{0.1in}}}{\textsl{What you know}} & \multicolumn{3}{c}{\textsl{What you want}}                                                                 \tabularnewline
 this...        & ... and this      &    sources                    &   symbols                          &    targets                     \tabularnewline
\midrule
\midrule %-------------------------------------------------------------------------------------------------------------------------------
  source        &  (nothing)        &      ---                      &  (none)                            &  getSuccessors(nwa, src)       \tabularnewline
                &  symbol           &      ---                      &        ---                         &  getSuccessors(nwa, src, sym)  \tabularnewline
                &  target           &      ---                      &  getSymbol(nwa, src, tgt, \&sym)   &   ---                          \tabularnewline
\midrule %-------------------------------------------------------------------------------------------------------------------------------
  symbol        &  (nothing)        &      ---                      &        ---                         &   (none)                       \tabularnewline
                &  target           & getPredecessors(nwa, sym, tgt)&       ---                         &   ---                          \tabularnewline
\midrule %-------------------------------------------------------------------------------------------------------------------------------
  target        &  (nothing)        & getPredecessors(nwa, tgt)     &  (none)                            &   ---                          \tabularnewline
\bottomrule\bottomrule
\end{tabular}
\end{threeparttable}
\label{Ta:query-all-transitions}
\end{sidewaystable}

\begin{sidewaystable}\sffamily
\caption{\textbf{Query functions for internal transitions.}
  These functions are in the namespace \texttt{wali::nwa::query};
  include the file \texttt{wali/nwa/query/internals.hpp}.
  A table entry of ``---'' means that combinations of arguments does not make
  sense.}
\begin{threeparttable}
\begin{tabular}{p{1in}p{1in}|@{\hspace{0.1in}}p{2.2in}p{2.2in}p{2in}}
\toprule\toprule
\multicolumn{2}{c|@{\hspace{0.1in}}}{\textsl{What you know}} & \multicolumn{3}{c}{\textsl{What you want}}                                                                 \tabularnewline
 this...        & ... and this      &    sources                    &   symbols                          &    targets                     \tabularnewline
\midrule
\midrule %-------------------------------------------------------------------------------------------------------------------------------
 (nothing)      &  (nothing)        & getSources(nwa)               &  getInternalSym(nwa)               &  getTargets(nwa)               \tabularnewline
\midrule %-------------------------------------------------------------------------------------------------------------------------------
  source        &  (nothing)        &      ---                      &  getInternalSym\_Source(nwa, src) \newline
                                                                       or getTargets(nwa, src)\RP        &  getTargets(nwa, src)\RP       \tabularnewline
                &  symbol           &      ---                      &        ---                         &  getTargets(nwa, src, sym)     \tabularnewline
                &  target           &      ---                      &  getInternalSym(nwa, src, tgt)     &   ---                          \tabularnewline
\midrule %-------------------------------------------------------------------------------------------------------------------------------
  symbol        &  (nothing)        & getSources\_Sym(nwa, sym)     &        ---                         &  getTargets\_Sym(nwa, sym)     \tabularnewline
                &  target           & getSources(nwa, sym, tgt)     &        ---                         &   ---                          \tabularnewline
\midrule %-------------------------------------------------------------------------------------------------------------------------------
  target        &  (nothing)        & getSources(nwa, tgt)\RP       &  getInternalSym\_Target(nwa, tgt) \newline
                                                                       or getSources(nwa, tgt)\RP        &   ---                          \tabularnewline
\bottomrule\bottomrule
\end{tabular}
\begin{tablenotes}
  \item[1] Returns a set of pairs (either source/symbol or symbol/target).
\end{tablenotes}
\end{threeparttable}
\label{Ta:query-internal-transitions}
\end{sidewaystable}

\begin{sidewaystable}\sffamily
\begin{threeparttable}
\caption{\textbf{Query functions for call transitions.} These functions are in the
  namespace \texttt{wali::nwa::query}; include the
  file \texttt{wali/nwa/query/calls.hpp}. The ``call site'' is the source of the transition (and uses the argument 
  name \texttt{call}), and the ``entry'' of the transition is the target (and
  uses the argument name \texttt{ent}). }
\begin{tabular}{p{1in}p{1in}|@{\hspace{0.1in}}p{2.2in}p{2.2in}p{2in}}
\toprule\toprule
\multicolumn{2}{c|@{\hspace{0.1in}}}{\textsl{What you know}} & \multicolumn{3}{c}{\textsl{What you want}}                                                                 \tabularnewline
 this...        & ... and this      &    call sites                   &   symbols                          &    entries                     \tabularnewline
\midrule
\midrule %-------------------------------------------------------------------------------------------------------------------------------
 (nothing)      &  (nothing)        & getCallSites(nwa)               &  getCallSym(nwa)                   &  getEntries(nwa)               \tabularnewline
\midrule %-------------------------------------------------------------------------------------------------------------------------------
 call site      &  (nothing)        &      ---                        &  getCallSym\_Call(nwa, call)\newline
                                                                         or getEntries(nwa, call)\RP       &  getEntries(nwa, call)\RP      \tabularnewline
                &  symbol           &      ---                        &        ---                         &  getEntries(nwa, call, sym)    \tabularnewline
                &  target           &      ---                        &  getCallSym(nwa, call, ent)        &   ---                          \tabularnewline
\midrule %-------------------------------------------------------------------------------------------------------------------------------
  symbol        &  (nothing)        & getCallSites\_Sym(nwa, sym)     &        ---                         &  getEntries\_Sym(nwa, sym)     \tabularnewline
                &  target           & getCallSites(nwa, sym, ent)     &        ---                         &   ---                          \tabularnewline
\midrule %-------------------------------------------------------------------------------------------------------------------------------
  entry         &  (nothing)        & getCallSites(nwa, ent)\RP       &  getCallSym\_Entry(nwa, ent) \newline
                                                                         or getCallSites(nwa, ent)\RP      &   ---                          \tabularnewline
\bottomrule\bottomrule
\end{tabular}
\begin{tablenotes}
  \item[1] Returns a set of pairs (either call site/symbol or symbol/entry).
\end{tablenotes}
\end{threeparttable}
\label{Ta:query-call-transitions}
\end{sidewaystable}

\begin{sidewaystable}\footnotesize\sffamily
\begin{threeparttable}
\caption{\textbf{Query functions for return transitions.} These functions are in the
  namespace \texttt{wali::nwa::query}; include the
  file \texttt{wali/nwa/query/returns.hpp}.   The ``exit site'' is the source of the transition
  (the first component) and uses the argument name \texttt{exit} in this table;
  the ``call predecessor'' is the second component and uses the argument
  name \texttt{call}; the symbol is the third component and uses the argument
  name \texttt{sym}; the ``return site'' is the fourth component and uses the
  argument name \texttt{ret}. }
\begin{tabular}{p{0.6in}p{0.65in}p{0.6in}|@{\hspace{0.1in}}p{1.75in}p{1.9in}p{1.9in}p{2in}}
\toprule\toprule
\multicolumn{3}{c|@{\hspace{0.1in}}}{\textsl{What you know}}  & \multicolumn{4}{c}{\textsl{What you want}}                                                                                                                                                \tabularnewline
 this...  & and this ...  & and this & exit sites & call predecessors &
 symbols & return sites \tabularnewline
\midrule
\midrule %--------------------------------------------------------------------------------------------------------------------------------------------------------------------------------------------------------------------
 (nothing)      &  (nothing)        &  (nothing)    & getExits(nwa)                 &  getCalls(nwa)                        &  getReturnSym(nwa)                           &  getReturns(nwa)                            \tabularnewline
\midrule %--------------------------------------------------------------------------------------------------------------------------------------------------------------------------------------------------------------------
 exit site      &  (nothing)        &  (nothing)    &      ---                      &  getCalls\_Exit(nwa, exit)\RP         &  getReturnSym\_Exit(nwa, exit) \newline
                                                                                                                               or getReturns\_Exit(nwa, exit)\RP\newline
                                                                                                                               or getCalls\_Exit(nwa, exit)\RP             &  getReturns\_Exit(nwa, exit)\RP             \tabularnewline
                \cline{2-7} %-------------------------------------------------------------------------------------------------------------------------------------------------------------------------------------------------
                &  call pred        &  (nothing)    &      ---                      &    ---                                &  getReturnSym\_ExitCall(nwa, exit, \newline
                                                                                                                               \phantom{getReturnSym\_ExitCall(}call) \newline
                                                                                                                               or getReturns(nwa, exit, call)\RP           &  getReturns(nwa, exit, call)\RP             \tabularnewline
                &                   &  symbol       &      ---                      &    ---                                &        ---                                   &  getReturns(nwa, exit, call, sym)           \tabularnewline
                &                   &  return       &      ---                      &    ---                                &  getReturnSym(nwa, exit, call, \newline
                                                                                                                               \phantom{getReturnSym(}ret)                 &    ---                                      \tabularnewline
                \cline{2-7} %-------------------------------------------------------------------------------------------------------------------------------------------------------------------------------------------------
                &  symbol           &  (nothing)    &      ---                      &  getCalls\_Exit(nwa, exit, sym)       &        ---                                   &  getReturns\_Exit(nwa, exit, sym)           \tabularnewline
                &                   &  return       &      ---                      &  getCalls(nwa, exit, sym, ret)        &        ---                                   &  getEntries(nwa, call, sym, ret)            \tabularnewline
                \cline{2-7} %-------------------------------------------------------------------------------------------------------------------------------------------------------------------------------------------------
                &  return site      &  (nothing)    &      ---                      &  getCalls(nwa, exit, ret)\RP          &  getReturnSym\_ExitRet(nwa, exit, \newline
                                                                                                                               \phantom{getReturnSym\_ExitRet(}ret) \newline
                                                                                                                               or getCalls(nwa, exit, ret)\RP              &   ---                                       \tabularnewline
\midrule %-------------------------------------------------------------------------------------------------------------------------------------------------------------------------------------------------------------------
 call pred      &  (nothing)        &  (nothing)    & getExits\_Call(nwa, call)\RP  &   ---                                 &  getReturnSym\_Call(nwa, call) \newline
                                                                                                                               or getReturns\_Call(nwa, call)\RP\newline
                                                                                                                               or getExits\_Call(nwa, call)\RP            &  getReturnSites(nwa, call) or \newline
                                                                                                                                                                             getCallSuccessors(nwa, call) or \newline
                                                                                                                                                                             getReturns\_Call(nwa, call)\RP              \tabularnewline
                \cline{2-7} %-------------------------------------------------------------------------------------------------------------------------------------------------------------------------------------------------
                &  symbol           &  (nothing)    & getExits\_Call(nwa, call, sym)&   ---                                 &        ---                                   &  getCallSuccessors(nwa, call, sym) \newline
                                                                                                                                                                              or getReturns\_Call(nwa, call, sym)        \tabularnewline
                &                   &  return       & getExits(nwa, call, sym, ret) &   ---                                 &        ---                                   &    ---                                      \tabularnewline
                \cline{2-7} %-------------------------------------------------------------------------------------------------------------------------------------------------------------------------------------------------
                &  return site      &  (nothing)    & getExits(nwa, call, ret)\RP   &   ---                                 &  getReturnSym\_CallRet(nwa, call, \newline
                                                                                                                               \phantom{getReturnSym\_CallRet(}ret) \newline
                                                                                                                               or getExits(nwa, call, ret)\RP              &   ---                                       \tabularnewline
\midrule %--------------------------------------------------------------------------------------------------------------------------------------------------------------------------------------------------------------------
 symbol         &  (nothing)        &  (nothing)    & getExits\_Sym(nwa, c)         &  getCalls\_Sym(nwa, c)                &   ---                                        &  getReturns\_Sym(nwa, sym)                  \tabularnewline
                \cline{2-7} %-------------------------------------------------------------------------------------------------------------------------------------------------------------------------------------------------
                &  return site      &  (nothing)    & getExits\_Ret(nwa, call, ret) &  getCallPredecessors(nwa, sym, ret) \newline
                                                                                       or getCalls\_Ret(nwa, sym, c)        &   ---                                        &   ---                                       \tabularnewline
\midrule %--------------------------------------------------------------------------------------------------------------------------------------------------------------------------------------------------------------------
 return site    &  (nothing)         & (nothing)    & getExits\_Ret(nwa, ret)       &  getCallPredecessors(nwa, ret) \newline
                                                                                       or getCalls\_Ret(nwa, ret)\RP        &  getReturnSym\_Ret(nwa, ret) or \newline
                                                                                                                               getCalls\_Ret(nwa, ret)\RP                  &   ---                                       \tabularnewline
\bottomrule\bottomrule
\end{tabular}
\begin{tablenotes}
  \item[1] Returns a set of pairs (a symbol with one of the states, in the order of the raw transition).
\end{tablenotes}
\end{threeparttable}
\label{Ta:query-return-transitions}
\end{sidewaystable}
\restoregeometry
