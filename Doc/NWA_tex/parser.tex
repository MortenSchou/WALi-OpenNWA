The NWA library contains an experimental parser for descriptions of
NWAs. The format is described below, and matches the result of calling
\texttt{NWA::print(std::ostream \&)} provided that the result of
calling \texttt{key2str} on the state and symbol keys produces strings
that match the description of \textsl{actual-names} in this grammar.

There are two functions which parse NWA descriptions:
\begin{itemize}
  \item\texttt{NWARefPtr read\_nwa(std::istream \&, std::string * name = NULL)}
  \item\texttt{std::map$<$std::string, NWARefPtr$>$ read\_nwa\_proc\_set(std::istream \&)}
\end{itemize}
A stream may contain descriptions of one or more automata; the first
function reads one, and the second function reads all.  (The return
type of the second function has a typedef name of
\texttt{ProcedureMap}.)

The former function reads the first description of an NWA from the
input string and returns it. Descriptions specify an optional name of
an NWA; if one is specified and \texttt{name} is non-null, then the
name is stored in the string pointed to by \texttt{name}.

The second function reads the entire input stream. It returns a map
from the name of each automaton specification to the parsed NWA. (This
form was originally created to read NWAs for an entire program, witch
each procedure in a separate NWA. The name of each NWA was the name of
the procedure in the original program.)

\section{NWA description format}

This section describes the grammar of the file format.

\textsl{Note:} Some characteristics in the following have the
following as part of their description: ``you should not depend on
this.'' In such cases, we reserve the right to change this behavior in
future versions and to not feel bad about it.

\textsl{Note:} This is experimental code; in particular, the
error-handling in it is basically non-existent. We detect when the
input doesn't match the grammar, but we signal such input failures
with assertion violations. In a future version of the library, we will
report errors more gracefully, almost certainly through exceptions.

\textsl{Note:} The NWA parser has unit tests embedded in the code. (It
does not use any unit test framework.) These tests can be run by
calling \texttt{wali::nwa::parser\_test\_all}. The file
\texttt{NwaParserMain.cpp} is a short program that does this.


The grammar for an NWA is as follows. In order to accept a couple
slightly different output formats we have used in the past, there are
some choices in whether braces are present and other aspects.


A single input stream can contain either a single NWA (just a series
of blocks) or multiple NWAs. If you would like to describe multiple
NWAs, each individual starts with the literal \textsl{nwa}.

  nwa-description  ::= ('nwa' name? ':'?)? '{'? block+ '}'?

 \textsl{Braces are required if 'nwa' is present and the name is not}
 to distinguish the first block header (\textsl{Q:}, \textsl{Q0:}, or
 \textsl{Qf:}) from the name of the NWA.

An NWA is a sequence of blocks; each block specifies one or more
states, symbols, or transitions.

  block  ::=  state-block     // lookahead = 'Q'
           |  sigma-block     // lookahead = 's'
           |  delta-block     // lookahead = 'd'

Blocks of the same type can be repeated; e.g. it is perfectly fine to
specify all transitions in one block, one block per transition, or
anywhere in between. The ``block header'' specifies what kind of block
it is, and the body is a comma-separated list of whatever entity the
block header specifies (e.g. states). The body may be surrounded by
curly braces, but they are not required unless the body is empty.

If a block specifies states, it can specify just that a state is
present, an initial state, or an accepting state; these are denoted by
the block headers \textsl{Q:}, \textsl{Q0:}, and \textsl{Qf:}
respectively.
 
  state-block  ::=  'Q:'   '{'?  name-list  '}'?
                 |  'Q0:'  '{'?  name-list  '}'?
                 |  'Qf:'  '{'?  name-list  '}'?

(It is not necessary to explicitly list all states; if a state is
listed in a transition, it is implicitly added to the machine if
needed. Thus it is quite possible to have a machine description with
no \textsl{Q:} blocks.)

A name-list is simply a comma-separate list of names of states.

  name-list ::=  (name (','  name)*)?

Recall that If the list is empty, it must be followed by a \textsl{\}}
(or EOF) to distinguish the next block header from the name of a
state.

The grammar for \textsl{name} will be specified below.


Symbol blocks are just like name blocks, except they specify symbol
names. The block header is simply 'sigma'.

  sigma-block  ::=  'sigma:'  '{'?  name-list  '}'?


A transition block can be for internal, call, or return transitions,
denoted by the block headers \textsl{delta\_i}, \textsl{delta\_c}, and
\textsl{delta\_r}, respectively.
 
  delta-block  ::=  'delta\_i:'  '{'?  triple-list  '}'?
                 |  'delta\_c:'  '{'?  triple-list  '}'?
                 |  'delta\_r:'  '{'?  quad-list    '}'?

The bodies in each case are simply a comma-separated list of triples
or quads:

  triple-list ::=  (triple  (','  triple)*)?
  quad-list   ::=  (quad  (','  quad)*)?

and each triple (resp., quad) is a 3-tuple (4-tuple) of names:

  triple  ::=  '('  name ','  name  ','  name  ')'
  quad  ::=  '('  name ','  name  ','  name  ','  name  ')'


All that remains is to define the grammar for `name'. Because the
format grew organically and we wished to keep compatibility with
existing log files (which could previously not be automatically parsed
but which we could look at), the `name' terminal is a bit convoluted.

The name consists of the \textsl{actual} name of the state or symbol
in question, followed by an optional parenthesis-delimited token which
is ignored. (In some of our output, the name is the result of
\texttt{key2str} and the optional token is the actual numeric value of
the key.)

  name  ::=  actual-name  dummy-token?

The actual-name portion generally behaves like a standard-programming
language identifier, but with one important difference: it can contain
balanced parentheses, and \textsl{any} characters are allowed within a
set of parentheses. (We actually allow a larger set of characters than
is typical, but you should not depend on this. Surround your names
with parentheses if you use ``special'' characters.)

We can now define the effective grammar. A `unit' is a letter or
parenthesis group.

  actual-name  ::=  unit+

  unit  ::=  normal-char  |  paren-group

  normal-char ::=  character other than whitespace, ',', or a paren

(\texttt{std::isspace} is used to determine whether a character is
  whitespace.)

  left-paren   ::=  '(' | '\{' | '[' | '$<$'
  right-paren  ::=  ')' | '\}' | ']' | '$>$'
  non-paren  ::=  a character other than any of the above

  paren-group ::=  left-paren  (non-paren | paren-group)*  right-paren

(Finally, we do not currently demand that the \textsl{types} of parens
  match up: e.g. \textsl{(1, 2, 3]} is a valid name. You should not
    depend on this feature.)
