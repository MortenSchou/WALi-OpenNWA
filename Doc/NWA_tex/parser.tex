\section{NWA serialization}
\label{Se:serialization}

It is possible to serialize an NWA in two different supported
formats.

The first is a description language of our own creation, described in
most of the rest of this section. We also provide a parser, with minor
caveats, for this format. The grammar the parser uses is defined
formally in \sectref{nwa-grammar}.

The second is output in Graphviz Dot format. The output is
natural. The color of an edge denotes the type of a transition: black
edges are internal transitions, green edges are call transitions, and
red edges are return transitions. Return transitions are labeled with
both the symbol for that transition and the call predecessor. (The
result of each Dot graph includes a key as a reminder for what the
colors mean.) Because it is possible to wind up with extremely long
key names (especially in a determinized NWA) that can essentially
destroy any ability of Dot to render useful output, by default any
state names longer than 20 characters are replaced by the numerical
value; this behavior can be changed however.

There is a member function which produces both types of output:
\begin{description}
  \item\texttt{std::ostream\& NWA::print(std::ostream \& stream) const}
    Writes the custom output described in this section
    to \texttt{stream}, returning \texttt{stream}.
    (This method is actually inherited from
    the \texttt{wali::Printable} abstract class.)

  \item\texttt{std::ostream\& NWA::print\_dot(std::ostream \& stream,
    std::string const \& name, bool abbreviate = true)} Outputs a Dot
    description of the NWA to the given stream. The name of the graph
    (\textsl{digraph "blah" \{...}) is given
    by \texttt{name}. Finally, if \texttt{abbreviate} is set
    to \texttt{false}, the abbreviation step above for state names is
    not done. Use this freedom at your peril.
\end{description}

\subsection{Parser}
\label{Se:parser}

The NWA library contains a semi-experimental parser for descriptions of
NWAs. The format is described below, and matches the result of calling
\texttt{NWA::print(std::ostream \&)}, provided that the result of
calling \texttt{key2str} on the state and symbol keys produces strings
that match the description of \textsl{actual-names} in this grammar.

\textsl{Note:} This is experimental code; in particular, the
error-handling in it is basically non-existent. We detect when the
input doesn't match the grammar, but we signal such input failures
with assertion violations. In a future version of the library, we will
report errors more gracefully, almost certainly through exceptions.

There are two functions which parse NWA descriptions:
\begin{itemize}
  \item\texttt{NWARefPtr read\_nwa(std::istream \&, std::string * name = NULL)}
  \item\texttt{std::map$<$std::string, NWARefPtr$>$ read\_nwa\_proc\_set(std::istream \&)}
\end{itemize}
A stream may contain descriptions of one or more automata; the first
function reads one, and the second function reads all.  (The return
type of the second function has a typedef name of
\texttt{ProcedureMap}.)

The former function reads the first description of an NWA from the
input string and returns it. Descriptions specify an optional name of
an NWA; if one is specified and \texttt{name} is non-null, then the
name is stored in the string pointed to by \texttt{name}.

The second function reads the entire input stream. It returns a map
from the name of each automaton specification to the parsed NWA. (This
form was originally created to read NWAs for an entire program, witch
each procedure in a separate NWA. The name of each NWA was the name of
the procedure in the original program.)

\subsection{Examples}

\figrefs{nwa-simple}{nwa-mattf} show examples of the NWA serialization format, illustrating
both the general form and some of the flexibilities in what precise
format is accepted.

\begin{figure}
  \centering
  \begin{tabular}{|p{1.75in}|}
    \verbatiminput{Figures/nwa-simple.nwa}
  \end{tabular}
  \caption{A simple NWA example}
  \label{Fi:nwa-simple}
\end{figure}

\begin{figure}
  \centering
  \begin{tabular}{|p{5.5in}|}
    \verbatiminput{Figures/nwa-mattf.nwa}
  \end{tabular}
  \caption {A more complex NWA}
  \label{Fi:nwa-mattf}
\end{figure}

\subsection{NWA description format}
\label{Se:nwa-grammar}

This section describes the grammar of the file format.

\textsl{Note:} Some characteristics in the following have the
following as part of their description: ``you should not depend on
this.'' In such cases, we reserve the right to change this behavior in
future versions and to not feel bad about it.

\textsl{Note:} The NWA parser has unit tests embedded in the code. (It
does not use any unit test framework.) These tests can be run by
calling \texttt{wali::nwa::parser\_test\_all}. The file
\texttt{NwaParserMain.cpp} is a short program that does this.


The grammar for an NWA is as follows. In order to accept a couple
slightly different output formats we have used in the past, there are
some choices in whether braces are present and other aspects.


A single input stream can contain either a single NWA (just a series
of blocks) or multiple NWAs. If you would like to describe multiple
NWAs, each individual starts with the literal \textsl{nwa}.

\newenvironment{grammar}{\begin{equation*}\begin{array}{lrl}}{\end{array}\end{equation*}}
\newcommand{\nonterm}[1]{\ensuremath{\langle}\textit{#1}\ensuremath{\rangle}}
\newcommand{\term}[1]{\textrm`\texttt{#1}\textrm'}
\newcommand{\cfgsp}{\ \ }

\begin{grammar}
  \nonterm{nwa-description} & ::= & \left(\term{nwa}\cfgsp\nonterm{name}?\cfgsp\term{:}?\right)?\cfgsp\term{\{}?\cfgsp\nonterm{block}{+}\cfgsp\term{\}}?
\end{grammar}

\textsl{Braces are required if 'nwa' is present and the name is absent}
 in order to
 distinguish the first block header (\textsl{Q:}, \textsl{Q0:},
 or \textsl{Qf:}) from the name of the NWA.

An NWA is a sequence of blocks; each block specifies one or more
states, symbols, or transitions.

\begin{grammar}
  \nonterm{block} &  ::= & \nonterm{state-block} \\
                  &    | & \nonterm{sigma-block} \\
                  &    | & \nonterm{delta-block}
\end{grammar}

Blocks of the same type can be repeated; e.g. it is perfectly fine to
specify all transitions in one block (as in \figref{nwa-mattf},
one block per transition (as in \figref{nwa-simple}), or
anywhere in between. The ``block header'' specifies what kind of block
it is, and the body is a comma-separated list of whatever entity the
block header specifies (e.g. states). The body may be surrounded by
curly braces, but they are not required unless the body is empty.

If a block specifies states, it can specify just that a state is
present, an initial state, or an accepting state; these are denoted by
the block headers \textsl{Q:}, \textsl{Q0:}, and \textsl{Qf:}
respectively.
 
\begin{grammar}
  \nonterm{state-block} &  ::= & \term{Q:} \cfgsp \term\{? \cfgsp \nonterm{name-list}\cfgsp\term\}? \\
                        &   |  & \term{Q0:}\cfgsp \term\{? \cfgsp \nonterm{name-list}\cfgsp\term\}? \\
                        &   |  & \term{Qf:}\cfgsp \term\{? \cfgsp \nonterm{name-list}\cfgsp\term\}?
\end{grammar}

(It is not necessary to explicitly list all states; if a state is
listed in a transition, it is implicitly added to the machine if
needed. Thus it is quite possible to have a machine description with
no \textsl{Q:} blocks.)

A name-list is simply a comma-separate list of names of states.

\begin{grammar}
  \nonterm{name-list} & ::= & (\nonterm{name}\cfgsp (\term{,}\cfgsp  \nonterm{name})*)?
\end{grammar}

Recall that If the list is empty, it must be followed by a \textsl{\}}
(or EOF) to distinguish the next block header from the name of a
state.

The grammar for \textsl{name} will be specified below.


Symbol blocks are just like name blocks, except they specify symbol
names. The block header is simply 'sigma'.

\begin{grammar}
  \nonterm{sigma-block} & ::= & \term{sigma:}\cfgsp  \term{\{}?\cfgsp  \nonterm{name-list}\cfgsp  \term{\}}?
\end{grammar}

A transition block can be for internal, call, or return transitions,
denoted by the block headers \textsl{delta\_i}, \textsl{delta\_c}, and
\textsl{delta\_r}, respectively.
 
\begin{grammar}
  \nonterm{delta-block} & ::= & \term{delta\_i:}\cfgsp  \term{\{}?\cfgsp  \nonterm{triple-list}\cfgsp  \term{\}}? \\
                        &   | & \term{delta\_c:}\cfgsp  \term{\{}?\cfgsp  \nonterm{triple-list}\cfgsp  \term{\}}? \\
                        &   | & \term{delta\_r:}\cfgsp  \term{\{}?\cfgsp  \nonterm{quad-list}  \cfgsp  \term{\}}?
\end{grammar}

The bodies in each case are simply a comma-separated list of triples
or quads:

\begin{grammar}
  \nonterm{triple-list} & ::= & (\nonterm{triple}\cfgsp  (\term{,}\cfgsp  \nonterm{triple})*)? \\
  \nonterm{quad-list}   & ::= & (\nonterm{quad}\cfgsp  (\term{,}\cfgsp  \nonterm{quad})*)?
\end{grammar}

and each triple (resp., quad) is a 3-tuple (4-tuple) of names:

\begin{grammar}
  \nonterm{triple} & ::= &  \term{(}\cfgsp  \nonterm{name}\cfgsp \term{,}\cfgsp  \nonterm{name}\cfgsp  \term{,}\cfgsp  \nonterm{name}\cfgsp  \term{)} \\
  \nonterm{quad} & ::= &  \term{(}\cfgsp  \nonterm{name}\cfgsp \term{,}\cfgsp  \nonterm{name}\cfgsp  \term{,}\cfgsp  \nonterm{name}\cfgsp  \term{,}\cfgsp  \nonterm{name}\cfgsp  \term{)}
\end{grammar}


All that remains is to define the grammar for `name'. Because the
format grew organically and we wished to keep compatibility with
existing log files (which could previously not be automatically parsed
but which we could look at), the `name' terminal is a bit convoluted.

The name consists of the \textsl{actual} name of the state or symbol
in question, followed by an optional parenthesis-delimited token which
is ignored. (In some of our output, the name is the result of
\texttt{key2str} and the optional token is the actual numeric value of
the key.)

\begin{grammar}
  \nonterm{name} & ::= &  \nonterm{actual-name}\cfgsp  \nonterm{dummy-token}?
\end{grammar}

The $\nonterm{actual-name}$ portion generally behaves like a standard-programming
language identifier, but with one important difference: it can contain
balanced parentheses, and \textsl{any} characters are allowed within a
set of parentheses. (We actually allow a larger set of characters than
is typical, but you should not depend on this. Surround your names
with parentheses if you use ``special'' characters.)

We can now define the effective grammar. A `unit' is a letter or
parenthesis group.

\begin{grammar}
  \nonterm{actual-name} & ::= &  \nonterm{unit}+ \\
&& \\
  \nonterm{unit} & ::= &  \nonterm{normal-char} \\
                 &   | &  \nonterm{paren-group} \\
&& \\
  \nonterm{normal-char}& ::= &  \textrm{character other than whitespace, \term{,}, or a paren}
\end{grammar}

(\texttt{std::isspace} is used to determine whether a character is
  whitespace.)

\begin{grammar}
  \nonterm{left-paren}  & ::= &  \term{(}\cfgsp |\cfgsp \term{\{}\cfgsp |\cfgsp \term{[}\cfgsp |\cfgsp \term{$<$} \\
  \nonterm{right-paren} & ::= &  \term{)}\cfgsp |\cfgsp \term{\}}\cfgsp |\cfgsp \term{]}\cfgsp |\cfgsp \term{$>$} \\
  \nonterm{non-paren} & ::= &  \textrm{a character other than any of the above} \\
&& \\
  \nonterm{paren-group}& ::= &  \nonterm{left-paren}\cfgsp  (\nonterm{non-paren}\ |\ \nonterm{paren-group})*\cfgsp  \nonterm{right-paren}
\end{grammar}

(Finally, we do not currently demand that the \textsl{types} of parens
  match up: e.g. \texttt{(1, 2, 3]} is a valid name. You should not
    depend on this feature.)
