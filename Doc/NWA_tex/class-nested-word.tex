\section{The \texttt{NestedWord} class}
\label{Se:class-nested-word}

The class \texttt{wali::nwa::NestedWord} is very simple. It provides support for
building nested words and iterating over their lengths; currently there
is no support for modifying them.

The representation used by this class is closer to that of a word in a
visibly-pushdown language~\cite{JACM:AM09}. It holds the linear contents of
a word, but does not store the nesting relation explicitly. Instead, each
position in the word is annotated with whether it is an internal, call, or
return position. The nesting relation is induced by the matchings between
calls and returns.

A position in the word is represented by the nested structure
\texttt{NestedWord::Position}; \texttt{Position} itself declares an
enumeration \texttt{Type} which is either \texttt{CallType},
\texttt{InternalType}, or \texttt{ReturnType}.
A \texttt{Position} object has two (public) fields: \texttt{Symbol symbol}
and \texttt{Position::Type type}; these hold the symbol at that position and
the type of the position.


The \texttt{NestedWord} class has just six functions:
\begin{description}
  \item\texttt{appendInternal(Symbol s)}
  \item\texttt{appendCall(Symbol s)}
  \item\texttt{appendReturn(Symbol s)}
    These append the symbol \texttt{s} to the linear word, and annotate that
    position as an internal, call, or return position, respectively.
  \item\texttt{append(Position p)}
    Appends \texttt{p} to the word.
  \item\texttt{begin()}
  \item\texttt{end()}
    These return a \texttt{NestedWord::const\_iterator} to the start or end
    of the nested word. The type returned by dereferencing these iterators is
    a \texttt{Position} object. (There is no non-const version of these
    functions.)
\end{description}

The only operation the library currently supports on
\texttt{NestedWord}s, besides building them, is to check whether a
nested word is in an NWA's language. This is done by the function
\texttt{wali::nwa::query::languageContains}; see
\sectref{query-language}.

