\section{The \texttt{wali::nwa::query} namespace}
\label{Se:namespace-query}

The \texttt{wali::nwa::query} namespace provides functions related to
querying both the structure and language of an automaton.

In this section, we first discuss how to retrieve information from an
automaton's transitions (\sectref{query-transitions}) then move on to other
queries about an automaton proper.
Following that, we discuss queries about the
\textsl{language} of an automaton (\sectref{query-language}).


\subsection{Querying information about an automaton's transitions}
\label{Se:query-transitions}
The \texttt{wali::nwa::query} namespace provides a large number of functions
for retrieving information about an \texttt{NWA}'s transitions. The questions
answered by these functions are of the form, e.g., ``what are all symbols that
appear on a return transition where the call predecessor state is \texttt{s1}
and the return site state is \texttt{s1}?''

Functions return information either about all transitions or about
transitions of a particular type (internal, call, or return). The names of
these functions have regular patterns based on the information that is
known and desired, but the rules for producing the names are perhaps not as
simple as they could be. Because of this,
Tabs.~\ref{Ta:query-all-transitions}--\ref{Ta:query-return-transitions} in \appref{Tables}
provide a quick reference for the functions.

The tables do not have \textsl{all} the information one needs to know in
order to call them, and the entries use some shorthands; but they provide enough
information for specifics to be
looked up in the corresponding header or Doxygen documentation. The caption of each table
gives the header that contains the functions listed. In the interest of
space, the types of the function arguments are omitted, but they are likely to
be what you expect. (The number of arguments is also always sufficient to
uniquely identify the function because all types are just \texttt{wali::Key}s
anyway.) Almost all functions return a \texttt{StateSet}, \texttt{SymbolSet},
or a \texttt{std::set} of pairs of a state and a symbol. (Those that return a
set of pairs are marked specially.)

%To use the table to answer the question above, we first look at the table for
%return transitions (\tableref{query-return-transitions}). We know the call
%predecessor and return site, so we find the corresponding row in the table
%(fourth from the bottom). We want to know all the symbols, so we look at the
%symbol column. We see there are two choices: \texttt{getReturnSym\_CallRet}
%and \texttt{getExits}. By looking at \texttt{returns.hpp} or Doxygen
%documentation, we can see that the former returns a \texttt{SymbolSet}, which
%is what we want. The latter returns
%a \texttt{std::set$<$std::pair$<$State,Symbol$>>$}, which contains the
%information we desire, but is likely to be less convenient to
%use. (The \texttt{State} component of each pair is the exit site, which is
%where the function name \texttt{getExits} comes from.)



%\newcolumntype{x}[1]{p{#1}}


\subsection{Querying other structural aspects of an automaton}
\label{Se:query-automaton}

The NWA library has two additional functions for querying attributes
of NWAs themselves. One computes whether an NWA is deterministic, and
the other computes whether two automata have any states in common.

These functions are declared in the header
\texttt{wali/nwa/query/automata.hpp} (and, of course, defined in the
namespace \texttt{wali::nwa::query}).

\begin{functionlist}
  \functionDef{bool}{isDeterministic}{NWA const \& nwa}{}
    Computes whether the given NWA is deterministic and returns the
    result. \textit{Warning:} The result is not cached in the NWA, and
    as presently-implemented, this function is very
    inefficient. Improvements to this will come in a later version of
    the library.

  \functionDefFirst{bool}{statesOverlap}{NWA const \& a, NWA const \& b}{}
    Computes whether the two NWAs share any states and returns the
    result.
\end{functionlist}


\subsection{Querying properties of an automaton's language}
\label{Se:query-language}

The NWA library has several functions for querying
properties of the language of an NWA (or how the languages of two NWAs
relate), and it also supports
the poststar and prestar queries used in program analysis.

The following functions perform standard language-theoretic queries. They are
located in namespace \texttt{wali::nwa::query} and are declared in
\texttt{wali/nwa/query/language.hpp}. They are:
\begin{functionlist}
  \functionDef{bool}{languageContains}{NWA const \& nwa, Nested Word
    const \& word}{}
    Computes whether \texttt{word} is a member of
    \texttt{nwa}'s language, and returns the result. (It simulates the
    NWA in a nondeterministic fashion; the NWA is not determinized to answer
    this query.) To a first approximation, the worst-case running time is
    $O(|\texttt{word}|\cdot w\log w\cdot\log n)$, where $w$ is the ``width'' of
    the nondeterminism of the NWA (i.e. the maximum number of
    simultaneous configurations that are possible when reading \texttt{word})
    and $n$ is the number of states in the NWA.

  \functionDefFirst{bool}{languageIsEmpty}{NWA const \& nwa}{}
    Computes
    whether the NWA's language is empty, and returns the result.

  \functionDefFirst{bool}{languageSubsetEq}{NWA const \& left, NWA const \& right}{}
    Computes whether $L(\texttt{left}) \subseteq L(\texttt{right})$
    and returns the result. Since this procedure must determinize \texttt{right},
    the worst-case running time is $O(2^{n^2}n^2m)$ where $n$ is the number of
    states in \texttt{right} and $m$ is the number of states in \texttt{left}.

  \functionDefFirst{bool}{languageEquals}{NWA const \& left, NWA const \& right}{}
    Equivalent to \texttt{languageSubsetEq(left, right) \&\&
    languageSubsetEq(right, left)}. This determinizes both machines; the
    running time is ... .
\end{functionlist}

There are also functions for performing a poststar and prestar
query on an NWA. These work by converting the NWA to a WPDS (see
\sectref{Conversions} and, in particular, \sectref{NWAtoPDS}) using
the function \texttt{NWAtoPDScalls} then running the query on the PDS.

Conceptually, the transition system queried by prestar and poststar is the space of
configurations of the NWA. A configuration is a pair $\langle q, s\rangle$,
where $q$ is the current state and $s$ is the stack of unmatched call
sites. (More exactly, it is the stack of states the machine was in before
reading a symbol in a call position that has not yet been matched.) In
reality,  the transition system is the configuration space of the WPDS
resulting from calling \texttt{NwaToWpdsCalls} (see
\sectrefs{Conversions}{wpds-forwards-flow-stacking-calls}); that space is
$\langle p, t\rangle$, where $p$ is a WPDS state and $t$ is the stack of
unmatched call sites with the NWA's ``current'' state $q$ added on top.

There are two variants of each function. One returns the WFA that results from
the query, and the other stores the WFA in an output parameter. The
\texttt{WFA} type is \texttt{wali::wfa::WFA}.

The functions that perform poststar and prestar are in the namespace
\texttt{wali::nwa::query} and are declared in the header
\texttt{wali/nwa/query/weighted.hpp}. They are:
\begin{functionlist}
  \functionDefEarly{WFA}{prestar}{NWA const \& nwa, WFA const \& input, WeightGen \& wg}{}
  \functionDefEarly{void}{prestar}{NWA const \& nwa, WFA const \& input, WFA \& output, WeightGen \& wg}{}
    Computes the prestar of the configurations specified by
    \texttt{input}, using the weights generated by \texttt{wg}. Either
    returns the result or stores it in the parameter \texttt{output}.

  \functionDefFirstEarly{WFA}{poststar}{NWA const \& nwa, WFA const \& input, WeightGen \& wg}{}
  \functionDefEarly{void}{poststar}{NWA const \& nwa, WFA const \& input, WFA \& output, WeightGen \& wg}{}
    Computes the poststar of the configurations specified by
    \texttt{input}, using the weights generated by \texttt{wg}. Either
    returns the result or stores it in the parameter \texttt{output}.
\end{functionlist}
